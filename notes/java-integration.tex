\section{``Our typestate language''}

Referring to the language formalised in Gay \etal \cite{gay10} and
implemented by Dimitris and Ornela.

In what follows, I use ``\TypestateJava'' to mean ``any Java-like
language with typestate''. I remain neutral about whether such
languages are a good idea; it's simply useful to have a name for the
concept.

\paragraph{Overall goals.}

There are a variety of possible research goals here. Here are some
possibilities, which I mention but do not propose; nor is the list
intended to be exhaustive or mutually exclusive.

\begin{itemize}
\item explore typestate as an OO paradigm without taking a Java-like
  language as a foundation;
\item implement \TypestateJava as a proof-of-concept, without worrying
  too much about compatibility;
\item design a 100\% backward-compatible \TypestateJava to provide a
  migration path and get some actual adoption;
\item demonstrate that \TypestateJava is convenient to use and
  provides a useful level of additional type-safety;
\item show how \TypestateJava can be used to build type-safe
  distributed or concurrent systems.
\end{itemize}

\noindent If we intend to have more examples that target the existing
language, we should also have a roadmap for developing the existing
language in the direction of the identified goals. For example, early
typestate language such as Plural \cite{bierhoff11} used pre- and
post-condition annotations on methods to specify state transitions.
More recent languages such as Plaid \cite{garcia14} unify states with
classes; in these ``typestate-oriented'' languages, state change is
object reclassification (avoiding the issue with multiple
inheritance).

\paragraph{Typestate definitions.}

We currently take a third approach, namely of introducing a separate
typestate definition. This may have merits, but it clearly has a
significant cost, namely the duplication of method and type
signatures. How it interacts with overloading is unclear (what if I
define two methods called \ttt{buy}?), as it what happens if there is
a mismatch between the typestate definition and the class definition (say .
The advantages/disadvantages of the approach need to be weighed in the
light of what we're trying to achieve.

\paragraph{Relationship between states and classes/interfaces.}

(A Java class cannot have two unrelated superclasses, so let's
restrict the discussion to interfaces.) An interface specifies the
methods available on an object. A typestate also does this, so a
natural question is: is a typestate an interface, annotated with
information about target states? Tracking the state of an object
statically is like implicitly casting it to a new interface; the
underlying object has to implement all the methods and all the
interfaces.

First we use a so-called \emph{annotation type} to define a new kind
of Java annotation called \ttt{Target}:

\begin{lstlisting}
import java.lang.annotation.*;

@Target(ElementType.METHOD)
public @interface Become {
   Class<? extends Object> value ();
}
\end{lstlisting}

\noindent This would allow us to write \ttt{@Become(State.class)} as
an annotation on a method, where \ttt{State} is the name of a state
interface. We can use these to decorate the interface methods with the
target state; think of this as a backwards-compatible way of extending
the syntax of Java to allow Plaid-style transition declarations
\cite{garcia14}.

\begin{lstlisting}
public interface Buyer {
   interface Init {
      @Become(Shop.class)
      void init (Channel u);
   }

   interface Shop {
       @Become(Shop.class)
       Price price (Product p);

       @ResultCase(is = Result.OK, then = @Become(Pay.class))
       @ResultCase(is = Result.ERROR, then = @Become(Shop.class))
       void buy (Product p);

       @Become(End.class)
       void stop ();
   }

   interface Pay {
       @Become(Shop.class)
       void pay (Payment p);
   }
}
\end{lstlisting}

\noindent The implementation is essentially as it was before, except
that the class \ttt{BuyerImpl} must implement each of the state
interfaces. The job of the typestate extensions to the compiler is to
allow the user to call methods on the state interfaces without having
to explicitly cast.

\begin{lstlisting}
class BuyerImpl implements Buyer.Init, Buyer.Shop, Buyer.Pay {
   public void init (Channel u) { ... }
   public void pay(Payment p) { ... }
   public Price price (Product p) { ... }
   public void buy (Product p) { ... }
   public void stop() { ... }
}
\end{lstlisting}

\noindent \ttt{Channel} would also be modelled as a class with its own
typestate interfaces.

\noindent The duplicate annotations like this are legal in Java 8.
Because annotation types can't make use of generics, the user would
also have to define
\begin{lstlisting}
@interface ResultCase {
   Result is ();
   Become then ();
}
\end{lstlisting}
